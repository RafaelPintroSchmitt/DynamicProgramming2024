\documentclass[11pt]{extarticle}
\usepackage{fullpage,amsmath,amsfonts,microtype,nicefrac,amssymb, amsthm}
\usepackage[left=1in, bottom=1in, top=1in, right = 1in]{geometry}
\usepackage{textcomp}
\usepackage{mathpazo}
\usepackage{mathrsfs}
\usepackage[T1]{fontenc}
\usepackage[utf8]{inputenc}
\usepackage[english]{babel}
\usepackage{graphicx}

\usepackage{microtype}

\usepackage{bm}
\usepackage{dsfont}
\usepackage{enumerate}
\usepackage{ragged2e}

\setlength{\parindent}{24pt}
\setlength{\jot}{8pt}


\usepackage[shortlabels]{enumitem}


%% FOOTNOTES
\usepackage[bottom]{footmisc}
\usepackage{footnotebackref}


%% FIGURE ENVIRONMENT
%\graphicspath{{}}
\usepackage[margin=15pt, font=small, labelfont={bf}, labelsep=period]{caption}
\usepackage{subcaption}
\captionsetup[figure]{name={Figure}, position=above}
\usepackage{float}
\usepackage{epstopdf}


%% NEW COMMANDS
\renewcommand{\baselinestretch}{1.25} 
\renewcommand{\qedsymbol}{$\blacksquare$}
\newcommand{\R}{\mathbb{R}}
\newcommand{\indep}{\mathrel{\text{\scalebox{1.07}{$\perp\mkern-10mu\perp$}}}}
\renewcommand{\b}{\begin}
\newcommand{\e}{\end}

%% NEWTHEOREM
\theoremstyle{plain}
\newtheorem{thm}{Theorem}
\newtheorem{lem}[thm]{Lemma}
\newtheorem{prop}[thm]{Proposition}

\theoremstyle{definition}
\newtheorem{defn}[thm]{Definition}
\newtheorem{ex}[thm]{Example}
\newtheorem{remark}[thm]{Remark}
\newtheorem{cor}[thm]{Corollary}

%% LINKS and COLORS
\usepackage[dvipsnames]{xcolor}
\usepackage{hyperref}
\definecolor{myred}{RGB}{163, 32, 45}
\hypersetup{
	%backref=true,
	%pagebackref=true,
	colorlinks=true,
	urlcolor=myred,
	citecolor=myred, 
	linktoc=all,     
	linkcolor=myred,
}

%% TABLE OF CONTENTS
\addto\captionsenglish{
	\renewcommand{\contentsname}
	{}% This removes the heading over the table of contents.
}



%%%%%%%%%%%%%%%%%%%%%%%%%%%%%%%%%%%%%%%%%%%%%%%%%%%%%%%%%%%%
%%%%%%%%%%%%%%%%%%%%%%%%%%%%%%%%%%%%%%%%%%%%%%%%%%%%%%%%%%%%
%%%%%%%%%%%%%%%%%%%%%%            END PREAMBLE           %%%%%%%%%%%%%%%%%%%%%%%%
%%%%%%%%%%%%%%%%%%%%%%%%%%%%%%%%%%%%%%%%%%%%%%%%%%%%%%%%%%%%
%%%%%%%%%%%%%%%%%%%%%%%%%%%%%%%%%%%%%%%%%%%%%%%%%%%%%%%%%%%%

\title{202A: Dynamic Programming and Applications\\[5pt] {\Large \textbf{Homework \#1}}}

\author{Andreas Schaab\footnote{
	UC Berkeley. Email: schaab@berkeley.edu.\\
	This course builds on the excellent teaching material developed, in particular, by David Laibson and Benjamin Moll. I would also like to thank Gabriel Chodorow-Reich, Pablo Kurlat, J\'on Steinsson, and Gianluca Violante, as well as QuantEcon, on whose teaching material I have drawn. 
}}

\date{}


\begin{document}

\maketitle


%%%%%%%%%%%%%%%%%%%%%%%%%%%%%%%%%%%%%%%%%%%%%%%%%%%%%%%%%%%%
%%%%%%%%%%%%%%%%%%%%%%%%%%%%%%%%%%%%%%%%%%%%%%%%%%%%%%%%%%%%
\section*{Problem 1: Discrete Time Markov Chains}

Time is discrete, with $t \in \{0, 1, ...\}$. Consider the stochastic process $\{ y_t \}$ which follows a Markov chain. Let $y_t$ denote the employment / earnings state of an individual in periods $t$. Consider the state space $y_t \in Y = \{ y^U, y^E \}$, where $y^U$ corresponds to unemployment and $y^E$ corresponds to employment. Let $\bm y$ denote the column vector $(y^U, y^E)'$ representing this state space (this is the grid you would construct on a computer). Suppose the employment dynamics of the individual are characterized by the invariant transition matrix 
\begin{equation*}
	P = \begin{pmatrix} \alpha & 1 - \alpha \\ 1 - \beta & \beta \end{pmatrix}.
\end{equation*}
We interpret a time period as a quarter. 

\vspace{5mm}
\noindent
\begin{enumerate}[(a)]
\item Give economic interpretations of $\alpha = P_{11}$ and $\beta = P_{22}$

\textbf{Answer:} $\alpha$ is the probability for an unemployed person to remain unemployed, $\beta$ is the probability that an employed person remains employed.
\item Why do the rows of $P$ sum to $1$?

\textbf{Answer:} Since there are two states, we have that the probability of an unemployed person remaining unemployed OR changing to employment must be one (analogously for employed). Therefore both rows must sum to 1.
\item Is there an absorbing state in this model?

\textbf{Answer:} Yes, P is a contraction, so there is a single fixed point.
\item Compute the probability of being unemployed two quarters after being employed. 

\textbf{Answer:} There are 2 paths: (1) Start employed, continue employed, become unemployed; (2) Start employed, become unemployed, remain unemployed.\\
The probability of (1) is $(\beta)\times (1-\beta)$ and the probability of (2) is $(1-\beta)\times (\alpha)$. So the probability of being unemployed two quarters after being employed is simply the sum of the two: 
$$(\beta)\times (1-\beta) + (1-\beta)\times (\alpha)$$
\item Denote the \textit{marginal (probability) distribution} of $y_t$ at time $t$ by $\psi_t$. $\psi_t(y^L)$ is the probability that process $y_t$ is in state $y^L$ at time $t$. It is easiest to think of $\psi_t$ as a time-varying row vector. Use the law of total probability to decompose $y_{t+1} = y^L$, accounting for all the possible ways in which state $y^L$ can be reached at time $t+1$. 

\textbf{Answer:} $y_{t+1}$ is a function of the probability of being in each state at $t$. In the simple binary case (unemployed and employed), this becomes:
$y_{t+1}[1] = P(U_{t+1}|U_{t})\times P(U_{t}) + P(U_{t+1}|E_{t})\times P(E_{t})$
Where $P(U_{t+1}|U_{t})$ is the probability of being unemployed at time $t+1$ given that you were unemployed at time $t$, and the result above is obtained by the law of total probability ($\{E{t},U{t} \}$ is a partition of the state space at time $t$). Similarly,
$y_{t+1}[2] = P(E_{t+1}|E_{t})\times P(E_{t}) + P(E_{t+1}|U_{t})\times P(U_{t})$.\\
Using the transition matrix, we can then find that: 
$y_{t+1}[1] = \alpha \times y_{t}[1] + (1-\alpha) \times y_{t}[2]$
$y_{t+1}[2] = (1-\beta) \times y_{t}[1] + (\beta) \times y_{t}[2]$

\item Show that the resulting equation can be written as the vector-matrix product
\begin{equation*}
	\psi_{t+1} = \psi_t P.
\end{equation*}
Therefore: The evolution of the marginal distribution of a Markov chain is obtained by post-multiplying by the transition matrix. 

\item Show that
\begin{equation*}
	X_0 \sim \psi_0 \implies X_t \sim \psi_0 P^t,
\end{equation*}
where $\sim$ reads ``is distributed according to''. 

\item We call $\psi^*$ a \textit{stationary distribution} of the Markov chain if it satisfies 
\begin{equation*}
	\psi^* = \psi^* P.
\end{equation*}
Compute the probability of being unemployed $n$ quarters after being employed. Take $n \to \infty$ and find the stationary distribution of this Markov chain. Find the stationary distribution by alternatively plugging into the above equation for $\psi^*$. 

\item Suppose $y_0 = y^H$. Solve for $\mathbb E_0 (y_t)$. Use the law of total / iterated expectation to relate expectation to probabilities. Then use the formulas for marginal (probability) distributions derived above. 
\end{enumerate}



%%%%%%%%%%%%%%%%%%%%%%%%%%%%%%%%%%%%%%%%%%%%%%%%%%%%%%%%%%%%
%%%%%%%%%%%%%%%%%%%%%%%%%%%%%%%%%%%%%%%%%%%%%%%%%%%%%%%%%%%%
\vspace{5mm}
\section*{Problem 2: Proof of the Contraction Mapping Theorem}

In class, we defined the Bellman operator $B$, which operates on functions $w$, and is defined by
\begin{equation*}
	(Bw)(x) \equiv \max_{x' \in \Gamma(x)} \bigg\{ F(x, x') + \beta w(x') \bigg\}
\end{equation*}
for all $x \in \mathcal X$ in the state space, where $\Gamma(x)$ is some constraint set---in our case, this was the budget constraint. The definition is expressed pointwise, but it applies to all possible values in the state space. We call $B$ an operator because it maps a function $w$ to a new function $Bw$. So both $w$ and $Bw$ map $\mathcal X$ into $\mathbb R$. Operator $B$ maps \textit{functions} and is therefore called a functional operator. In class, we showed that the solution of the Bellman equation---the value function---is a fixed point of the Bellman operator.

What does it mean to \textit{iterate} $B^n w$?
\begin{align*}
	(Bw) (x) &= \max_{x' \in \Gamma(x)} \bigg\{ F(x, x') + \beta w(x') \bigg\} \\
	(B(Bw)) (x) &= \max_{x' \in \Gamma(x)} \bigg\{ F(x, x') + \beta (Bw)(x') \bigg\} \\
	(B(B^2w)) (x) &= \max_{x' \in \Gamma(x)} \bigg\{ F(x, x') + \beta (B^2w)(x') \bigg\} \\
	\vdots & \vdots  \\
	(B(B^nw)) (x) &= \max_{x' \in \Gamma(x)} \bigg\{ F(x, x') + \beta (B^nw)(x') \bigg\}.
\end{align*}
What does it mean for functions to converge to a limiting function? Let $v_0$ be some guess for the value function, then convergence would mean
\begin{equation*}
	\lim_{n \to \infty} B^n v_0 = v.
\end{equation*}
And why might $B^n w$ converge as $n \to \infty$? The answer is that $B$ is a \textit{contraction mapping}.


\vspace{5mm}
\begin{defn}
	
	Let $(S, d)$ be a metric space and $B: S \to S$ be a function that maps $S$ into intself. $B$ is a contraction mapping if for some $\beta \in (0, 1)$, $d(Bf, Bg) \leq \beta d(f, g)$, for any two functions $f$ and $g$.\footnote{
		A metric $d$ is a way of representing the distance between two functions, or two members of (metric) space $S$. One example: the supremum pointwise gap.
	}
	
\end{defn}


\vspace{5mm}
\noindent
Intuitively, $B$ is a contraction mapping if applying the operator $B$ to any two functions $f$ and $g$ (that are not the same) moves them strictly closer together. $Bf$ and $Bg$ are strictly closer together than $f$ and $g$. We can now state the contraction mapping theorem. 


\vspace{5mm}
\begin{thm}

	If $(S, d)$ is a complete metric space and $B: S \to S$ is a contraction mapping, then: 
	\begin{enumerate}[(i)]
		\item $B$ has exactly one fixed point $v \in S$
		\item For any $v_0 \in S$, $\lim_{n \to \infty} B^n v_0 = v$
		\item $B^n v_0$ has an exponential convergence rate at least as great as $- \ln(\beta)$
	\end{enumerate}
	
\end{thm}


\vspace{5mm}
\noindent
In this problem, we will illustrate and prove the contraction mapping theorem.

\begin{enumerate}[(a)]
\item Consider the contraction mapping $(Bw)(x) \equiv h(x) + \alpha w(x)$ with $\alpha \in (0, 1)$. Iteratively apply the operator $B$ and show that 
\begin{equation*}
	\lim_{n \to \infty} (B^n f)(x) = \frac{h(x)}{1-\alpha}
\end{equation*}
Argue that this shows that the fixed point of this operator $B$ is consequently the function $v(x) = \frac{1}{1-\alpha} h(x)$. Show that $(Bv)(x) = v(x)$. 

\item \textbf{Optional:} Now we will prove the contraction mapping theorem in 3 steps (we will not prove the convergence rate). Show that $\{ B^n f_0\}_{n=0}^\infty$ is a Cauchy sequence. (Cauchy sequence definition: Fix any $\epsilon > 0$. Then there exists $N$ such that $d(B^m f_0, B^n f_0) \leq \epsilon$ for all $m, n \leq N$.) 

\item \textbf{Optional:} Show that the limit point $v$ is a fixed point of $B$. 

\item \textbf{Optional:} Show that only one fixed point exists. 

\end{enumerate}


%%%%%%%%%%%%%%%%%%%%%%%%%%%%%%%%%%%%%%%%%%%%%%%%%%%%%%%%%%%%
%%%%%%%%%%%%%%%%%%%%%%%%%%%%%%%%%%%%%%%%%%%%%%%%%%%%%%%%%%%%
\vspace{5mm}
\section*{Problem 3: Blackwell's Sufficiency Conditions}

We now show that there are in fact sufficient conditions for an operator to be contraction mapping.

\begin{thm}[Blackwell's Sufficient Conditions]
	
	Let $X \subset \mathbb R^l$ and let $C(X)$ be a space of bounded functions $f:X \to \mathbb R$, with the sup-metric. Let $B : C(X) \to C(X)$ be an operator satisfying two conditions:
	\begin{enumerate}[1.]
		\item monotonicity: if $f,g\in C(X)$ and $f(x)\leq g(x)$ $\forall x\in X$,%
		\newline
		then $(Bf)(x)\leq (Bg)(x),$ $\forall x\in X$\newline
		
		\item discounting: there exists some $\delta \in (0,1)$ such that 
		\[
		\lbrack B(f+a)](x)\leq (Bf)(x)+\delta a\text{ \ }\forall \text{ }f\in C(X),\
		a\geq 0,\ x\in X. 
		\]%
	\end{enumerate}
	
	\noindent
	Then, $B$ is a contraction with modulus $\delta$.

\end{thm}

\noindent
Note that $a$ is a constant and $(f+a)$ is the function generated by adding $a$ to the function $f$. Blackwell's conditions are sufficient but not necessary for $B$ to be a contraction.

\vspace{5mm}
\noindent
In this problem, we will prove these sufficient conditions.
\begin{enumerate}[(a)]
\item Let $d$ be the sup-metric and show that, for any $f, g \in C(X)$, we have $f(x) \leq g(x) + d(f, g)$ for all $x$

\item Use monotonicity and discounting to show that, for any $f, g \in C(X)$, we have $(Bf)(x) \leq (Bg)(x) + \delta d(f, g)$ and $(Bg)(x) \leq (Bf)(x) + \delta d(f, g)$. 

\item Combine these to show that $d(Bf, Bg) \leq \delta d(f, g)$. 
\end{enumerate}



%%%%%%%%%%%%%%%%%%%%%%%%%%%%%%%%%%%%%%%%%%%%%%%%%%%%%%%%%%%%
%%%%%%%%%%%%%%%%%%%%%%%%%%%%%%%%%%%%%%%%%%%%%%%%%%%%%%%%%%%%
\vspace{5mm}
\section*{Problem 4: Example of Blackwell's Conditions}

We will now work out a simple example to illustrate these sufficient conditions. In particular, consider the Bellman operator in a consumption problem with stochastic asset returns, stochastic labor income, and a liquidity constraint:
\begin{equation*}
	(Bf)(x)=\sup_{c\in \lbrack 0,x]}\left\{ u(c)+\delta \mathbb Ef(\tilde{R}_{+1}(x-c)+
	\tilde{y}_{+1})\right\} \text{ \ }\forall x\text{\ } 
\end{equation*}
Notionally, $\tilde R$ and $\tilde y$ just underscore that these are random variables. The $_{+1}$ subscript underscores that these random variables are realized next period (in class, we used $'$ for this). The liquidity constraint is encoded in $c \in [0, x]$. (Why?)


\vspace{5mm}
\noindent
\begin{enumerate}[(a)]
\item Interpret each term in the definition of this Bellman operator.

\item Explicitly write out the budget constraint that is used here.

\item Check the first of Blackwell's conditions: monotonicity.

\item Check the second of Blackwell's conditions: discounting.
\end{enumerate}



%%%%%%%%%%%%%%%%%%%%%%%%%%%%%%%%%%%%%%%%%%%%%%%%%%%%%%%%%%%%
%%%%%%%%%%%%%%%%%%%%%%%%%%%%%%%%%%%%%%%%%%%%%%%%%%%%%%%%%%%%
\vspace{5mm}
\section*{Problem 5: Neoclassical Growth Model with Log Utility}

Recall the neoclassical growth model we discussed in class. Assuming log utility, full depreciation, and a decreasing-returns production function, preferences can be written as
\begin{equation*}
	V(k_0) = \max_{ \{ k_{t+1} \}_{t=0}^\infty} \sum_{t=0}^\infty \beta^t \log (k_t^\alpha - k_{t+1})
\end{equation*}
where $0 < \alpha < 1$, subject to the constraint
\begin{equation*}
	k_{t+1} \in [0, k_t^\alpha] \equiv \Gamma(k_t).
\end{equation*}
Think of $k_t^\alpha$ as the resources you have available, and so the most you would be allowed to save is $k_t^\alpha$. We represent this constraint by the \textit{feasibility set} $\Gamma(k_t)$. (This is the more general notation you will find in Stokey-Lucas, for example.)

Consider the associated Bellman equation 
\begin{equation*}
	V(k) = \max_{k' \in \Gamma(k)} \bigg\{ \ln (k^\alpha - k') + \beta V(k') \bigg\}.
\end{equation*}

\begin{enumerate}[(a)]
\item Try to solve the Bellman equation by \textit{guessing} a solution. Specifically, start by guessing that the form of the solution is 
\begin{equation*}
	V(k) = \psi + \phi \log k.
\end{equation*}
We will solve for the coefficients $\psi$ and $\phi$, and show that $V(k)$ solves the functional equation. Rewrite the functional equation substituting in $V(k) = \psi + \phi \log k$. Use the Envelope Theorem (ET) and the First Order Condition (FOC) to show 
\begin{equation*}
	\phi = \frac{\alpha}{1 - \alpha \beta}.
\end{equation*}
Now use the FOC to show 
\begin{equation*}
	k'(k) = \alpha \beta k^\alpha.
\end{equation*}
Finally, show that the functional equation is satisfied at all feasible values of $k_0$ if 
\begin{equation*}
	\psi =\frac{1}{1-\beta }\bigg[ \log (1-\alpha \beta)+\frac{\alpha \beta}{1-\alpha \beta}\log (\alpha \beta )\bigg] .
\end{equation*}
You have now solved the functional equation by using the guess and verify method.


\item We have derived the policy function
\begin{equation*}
	k'(k) = g(k) = \alpha \beta k^\alpha.
\end{equation*}
Derive the optimal sequence of state variables $\{k_t^*\}_{t=0}^\infty$ which would be generated by this policy function. Show that 
\begin{equation*}
	V(k_0) = \sum_{t=0}^\infty \beta^t \log \bigg( (k_t^*)^\alpha - k_{t+1}^* \bigg),
\end{equation*}
thereby confirming that this policy function is optimal.


\item Consider the Bellman (functional) operator $B$ defined by
\begin{equation*}
	(B f)(k) = \sup_{k' \in \Gamma(k)} \Big\{ \log (k^\alpha - k') + \beta f(k') \Big\}.
\end{equation*}
Let $\hat V(k) = \frac{\alpha \log k}{1 - \alpha \beta}$. Show that
\begin{equation*}
	(B^n \hat V)(k) = \frac{1-\beta^n}{1-\beta} \bigg[\log (1-\alpha \beta) + \frac{\alpha \beta}{1 - \alpha \beta} \log (\alpha \beta) \bigg] + \frac{\alpha \log k}{1 - \alpha \beta}.
\end{equation*}
To prove this you'll need to show that $k' = \alpha \beta k^\alpha$, and substitute this expression into the functional operator. Let,
\begin{equation*}
	\lim_{n \rightarrow \infty} (B^n \hat V)(k) = V(k)
\end{equation*}
Confirm that $V(k)$ is the same solution to the the functional equation that you derived in part (a). You have now solved the functional equation by iterating the operator $T$ on a starting guess.

\end{enumerate}



%%%%%%%%%%%%%%%%%%%%%%%%%%%%%%%%%%%%%%%%%%%%%%%%%%%%%%%%%%%%
%%%%%%%%%%%%%%%%%%%%%%%%%%%%%%%%%%%%%%%%%%%%%%%%%%%%%%%%%%%%
\vspace{10mm}
\section*{Problem 6: A Model with Equity}

\vspace{5mm}
\noindent
Assume that a consumer with only equity wealth must choose period by period consumption in a discrete-time dynamic optimization problem. Specifically, consider the sequence problem
\begin{equation*}
	V(x_0) = \max_{ \{ c_t \}_{t=0}^\infty } \mathbb E_0 \sum_{t=0}^\infty e^{- \rho t} u(c_t)
\end{equation*}
subject to the constraints
\begin{equation*}
	x_{t+1} = e^{r + \sigma u_t - \frac{\sigma^2}{2}} (x_t - c_t)
\end{equation*}
where $u_t$ is iid and $u_t \sim \mathcal N(0, 1)$. There is a feasibility constraint $c_t \in [0, x_t]$. And we assume an endowment $x_0 > 0$. Here, $x_t$ represents equity wealth at period $t$ and $c_t$ is consumption in period $t$. The consumer has discount rate $\rho$. The consumer can only invest in a risky asset with expected return $e^r = \mathbb E e^{r + \sigma u_t - \frac{\sigma^2}{2}}$. And we assume CRRA preferences with $u(c) = \frac{1}{1-\gamma} c^{1-\gamma}$, with $\gamma \in [0, \infty]$. We call this \textit{constant} relative risk aversion because the relative risk aversion coefficient
\begin{equation*}
	- \frac{c u''(c)}{u'(c)} = \gamma
\end{equation*}
is constant. Notice that CRRA consumption preferences are \textit{homothetic}, and this is what allows us to analytically solve this problem. 

The associated Bellman equation is 
\begin{equation*}
	V(x) = \max_{x' \in [0, x]} \bigg\{ u(x - x') + \mathbb E e^{- \rho} V \bigg( e^{r + \sigma u - \frac{\sigma^2}{2}} x' \bigg) \bigg\}.
\end{equation*}


\begin{enumerate}[(a)]
\item Explain all terms in this Bellman equation. Why is $u$ not a state variable, i.e., why don't we have $V(x, u)$? Make sure that this equation makes sense to you.


\item Now guess that the value function takes the special form
\begin{equation*}
	V(x) = \phi \frac{x^{1-\gamma}}{1-\gamma} .
\end{equation*}
Note the close similarity between this functional form and the functional form of the utility function. Assuming that the value function guess is correct, use the Envelope Theorem to derive the consumption function:
\begin{equation*}
c=\phi^{-\frac{1}{\gamma}} x .
\end{equation*}
Now verify that the Bellman Equation is satisfied for a particular value of $\phi$. Do not solve for $\phi$ (it's a very nasty expression). Instead, show that
\begin{equation*}
	\log (1 - \phi^{-\frac{1}{\gamma}}) = \frac{1}{\gamma} \Big[ (1-\gamma) r - \rho \Big] + \frac{1}{2}(\gamma-1) \sigma^2.
\end{equation*}


\item Now consider the $\log$ of the ratio of $c_{t+1}$ and $c_t$. Show that
\begin{equation*}
	\mathbb E \log \bigg(\frac{c_{t+1}}{c_t}\bigg) = \frac{1}{\gamma}(r-\rho) + \frac{\gamma}{2} \sigma^2 - \sigma^2.
\end{equation*}


\item Interpret the previous equation for the certainty case $\sigma = 0$. Note that $\log (\frac{c_{t+1}}{c_t}) = \log c_{t+1} - \log c_t$ is the growth rate of consumption. Explain why $\log c_{t+1} - \log c_t$ increases in $r$ and decreases in $\rho$. Why does the coefficient of relative risk aversion $\gamma$ appear in the denominator of the expression? Why does the coefficient of relative risk aversion regulate the consumer's willingness to substitute consumption between periods?

\end{enumerate}




%%%%%%%%%%%%%%%%%%%%%%%%%%%%%%%%%%%%%%%%%%%%%%%%%%%%%%%%%%%%
%%%%%%%%%%%%%%%%%%%%%%%%%%%%%%%%%%%%%%%%%%%%%%%%%%%%%%%%%%%%
\vspace{5mm}
\section*{Problem 7: Optimal Stopping}

Consider the optimal stopping application from class: Each period $t = 0, 1, \ldots$ the consumer draws a job offer from a uniform distribution with support in the unit interval: $x \sim \text{unif}[0, 1]$. The consumer can either accept the offer and realize net present value $x$, or the consumer can wait another period and draw again. Once you accept an offer the game ends. Waiting to accept an offer is costly because the value of the remaining offers declines at rate $\rho = - \log(\beta)$ between periods. The Bellman equation for this problem is:
\begin{equation*}
	V(x) = \max \bigg\{ x, \; \beta \mathbb E V(x') \bigg\}
\end{equation*}
where $x'$ is your next draw, which is a random variable.


\vspace{2mm}
\begin{enumerate}[(a)]
\item Explain the intuition behind this Bellman equation. Explain every term.

\item Consider the associated functional operator:
\begin{equation*}
	(Bw)(x) = \max \bigg\{ x, \; \beta \mathbb E w(x') \bigg\}
\end{equation*}
for all $x$. Using Blackwell's conditions, show that this Bellman operator is a contraction mapping. 

\item What does the contraction mapping property imply about $\lim_{n \to \infty} B^n w$, where $w$ is \textit{any} arbitrary function? 

\item Suppose we make a (bad?) guess $w(x) = 1$ for all $x$. Analytically iterate on $B^n w$ and show that 
\begin{equation*}
	\lim_{n \to \infty} (B^n w) (x) = V(x) = \begin{cases}
		x^* & \text { if } x \leq x^* \\
		x & \text { if } x > x^*
	\end{cases}
\end{equation*}
where
\begin{equation*}
	x^* = e^\rho \bigg( 1 - \Big[ 1 - e^{- 2 \rho} \Big]^\frac{1}{2} \bigg).
\end{equation*}
\end{enumerate}



%%%%%%%%%%%%%%%%%%%%%%%%%%%%%%%%%%%%%%%%%%%%%%%%%%%%%%%%%%%%
%%%%%%%%%%%%%%%%%%%%%%%%%%%%%%%%%%%%%%%%%%%%%%%%%%%%%%%%%%%%
\vspace{5mm}
\section*{Problem 8: Optimal Project Completion}

Every period you draw a cost $c$ distributed uniformly between $0$ and $1$ for completing a project. If you undertake the project, you pay $c$, and complete the project with probability $1-p$. Each period in which the project remains uncompleted, you pay a late fee of $l$. The game continues until you complete the project.


\vspace{2mm}
\begin{enumerate}[(a)]
\item Write down the Bellman Equation assuming no discounting. Why is it ok to assume no discounting in this problem?

\item Derive the optimal threshold: $c^* = \sqrt{2l}$. Explain intuitively, why this threshold does not depend on the probability of failing to complete the project, $p$. 

\item How would these results change if we added discounting to the framework? Redo steps a and b, assuming that the agent discounts the future with discount factor $0 < \beta < 1$ and assuming that $p = 0$. Show that the optimal threshold is given by
\begin{equation*}
	c^* = \frac{1}{\beta} \bigg( \beta - 1 + \sqrt{ (1-\beta)^2 + 2 \beta^2 l } \bigg)
\end{equation*}

\item When $0 < \beta < 1$, is the optimal value of $c^*$ still independent of the value of $p$? If not, how does $c^*$ qualitatively vary with $p$? Provide an intuitive argument.

\item Take the perspective of an agent who has not yet observed the current period's draw of $c$. Prove that the expected delay until completion is given by:
\begin{equation*}
	\frac{1}{c^*(1-p)} - 1
\end{equation*}

\end{enumerate}


\end{document}











