\documentclass[11pt, aspectratio=169]{beamer}

\usepackage{amsmath, amsfonts, microtype, nicefrac, amssymb, amsthm, centernot}

\usepackage{pgfpages}

\usepackage{helvet}
\usepackage[default]{lato}
\usepackage{array}

\usefonttheme[onlymath]{serif}

\usepackage[utf8]{inputenc}
\usepackage[T1]{fontenc}
\usepackage{textcomp}
\usepackage{bm}

\usepackage{mathpazo}
\usepackage{hyperref}
\usepackage{multimedia}
\usepackage{graphicx}
\usepackage{multirow}
\usepackage{graphicx}
\usepackage{dcolumn}
\usepackage{bbm}
\newcolumntype{d}[0]{D{.}{.}{5}}

\usepackage{graphicx}
\usepackage[space]{grffile}
\usepackage{booktabs}

\usepackage{setspace}

\usepackage{transparent}


%%% FIGURES %%%
\usepackage{caption, subcaption}
\usepackage{booktabs, siunitx}
\usepackage{pgfplots} 
%\usepackage[outdir=./figures]{epstopdf}
\usepackage{float}
\usepackage{graphicx}
\usepackage[absolute, overlay]{textpos}
\usepackage{epstopdf}


%%% TIKZ %%%
\usepackage{tikz}
\usepackage{verbatim}
\usetikzlibrary{arrows.meta}
\usetikzlibrary{positioning}
\usetikzlibrary{bending}
\usetikzlibrary{snakes}
\usetikzlibrary{calc}
\usetikzlibrary{arrows}
\usetikzlibrary{decorations.markings}
\usetikzlibrary{shapes.misc}
\usetikzlibrary{matrix, shapes, arrows, fit, tikzmark}


%%% ALGORITHM %%%
\usepackage{algorithm}
\usepackage[noend]{algpseudocode}
\usepackage{multimedia}


%%% APPENDIX SLIDE NUMBERING %%%
\usepackage{appendixnumberbeamer}


%%% BEAMER BUTTON %%%
%\setbeamertemplate{button}{\tikz
	%\node[
	%	inner xsep = 2pt, 
	%	draw = structure!0, 
	%	fill = myblue, 
	%	rounded corners = 4pt]{\color{white} \tiny\insertbuttontext};
	%}


%%% COLORS %%%
\definecolor{blue}{RGB}{0,38,118}
\definecolor{red}{RGB}{213,94,0}
\definecolor{yellow}{RGB}{240,228,66}
\definecolor{green}{RGB}{0,158,115}

\definecolor{myred}{RGB}{163,32,45}
\definecolor{navyblue}{rgb}{0.05,0.2,0.70}
\definecolor{myblue}{RGB}{0,51,150}
\definecolor{myorange}{RGB}{255,140,0}
\definecolor{myref}{RGB}{160,160,160}
\definecolor{shock}{RGB}{0, 125, 34}%{50, 168, 82}

\definecolor{background}{RGB}{255,253,218}

% Define a new transparent color
\definecolor{trans}{rgb}{1,1,1}
\colorlet{trans}{black!20} % 0 percent opacity

\hypersetup{
  colorlinks=false,
  linkbordercolor = {white},
  linkcolor = {blue}
}

\setbeamercolor{frametitle}{fg=blue}
\setbeamercolor{title}{fg=black}
\setbeamertemplate{footline}[frame number]
\setbeamertemplate{navigation symbols}{} 
\setbeamertemplate{itemize items}{-}
\setbeamercolor{itemize item}{fg=blue}
\setbeamercolor{itemize subitem}{fg=blue}
\setbeamercolor{enumerate item}{fg=blue}
\setbeamercolor{enumerate subitem}{fg=blue}
\setbeamercolor{button}{bg=background, fg=blue,}

%\setbeamercolor{background canvas}{bg=background}


%%% FRAME TITLE %%%
\setbeamerfont{title}{series=\bfseries, parent=structure}
\setbeamerfont{frametitle}{series=\bfseries, parent=structure}


%%% TRANSITION FRAME %%%
\newenvironment{transitionframe}{
	\setbeamercolor{background canvas}{bg=blue}
	\begin{frame}
		\thispagestyle{empty}
		\addtocounter{framenumber}{-1}
		\vspace{42mm}
		\hspace{4mm} }{
		\begin{tikzpicture}
			\tikz \fill [white] (1,6) rectangle (20,10);
		\end{tikzpicture}
	\end{frame}
}


%%% OUTLINE %%%
\AtBeginSection[]
{
	\begin{frame}
       \frametitle{Roadmap of Talk}
       \tableofcontents[currentsection]
   \end{frame}
}
\setbeamercolor{section in toc}{fg=blue}
\setbeamercolor{subsection in toc}{fg=red}
\setbeamersize{text margin left=1em,text margin right=1em} 


%%% ENVIRONMENTS
\newenvironment{witemize}{\itemize\addtolength{\itemsep}{10pt}}{\enditemize}

\makeatother
\setbeamertemplate{itemize items}{\large\raisebox{0mm}{\textbullet}}
\setbeamertemplate{itemize subitem}{\footnotesize\raisebox{0.15ex}{--}}
\setbeamertemplate{itemize subsubitem}{\Tiny\raisebox{0.7ex}{$\blacktriangleright$}}

\setbeamertemplate{enumerate item}[default]
\setbeamertemplate{enumerate subitem}{\textbullet}
\makeatletter

% ITEMIZE SPACING:
% \usepackage{xpatch}
% \xpatchcmd{\itemize}
% {\def\makelabel}
% {\setlength{\itemsep}{0mm}\def\makelabel}
% {}
% {}


%%% PRETTY ENUMERATE %%%
% \usepackage{stackengine,xcolor}
% \newcommand\circnum[2]{\stackinset{c}{}{c}{.1ex}{\small\textcolor{white}{#2}}%
	% 	{\abovebaseline[-.7ex]{\Huge\textcolor{#1}{$\bullet$}}}}
% \newenvironment{myenum}
% {\let\svitem\item
	% 	\renewcommand\item[1][black]{%
		% 		\refstepcounter{enumi}\svitem[\circnum{##1}{\theenumi}]}%
	% 	\begin{enumerate}}{\end{enumerate}}
\usepackage{stackengine,xcolor,graphicx}
\newcommand\circnum[2]{\smash{\stackinset{c}{}{c}{.2ex}{\small\textcolor{white}{#2}}%
		{\abovebaseline[-1.1ex]{\Huge\textcolor{#1}{\scalebox{1.5}{$\bullet$}}}}}}
\newenvironment{myenum}
{\let\svitem\item
	\renewcommand\item[1][black]{%
		\refstepcounter{enumi}\svitem[\circnum{##1}{\theenumi}]}%
	\begin{enumerate}}{\end{enumerate}}

\newcommand{\notimplies}{\;\not\!\!\!\implies}



%%%%%%%%%%%%%%%%%%%%%%%%%%  TITLE   %%%%%%%%%%%%%%%%%%%%%%%%%%%%%%%%
\title[]{\\[8pt]
	{\large \color{blue} Dynamic Programming and Applications \\[5pt] \normalfont{Investment} \\[10pt] \normalfont{Lecture 9}}}
\author[Schaab]{Andreas Schaab}
\institute{}
\subject{}
\date{}



%%%%%%%%%%%%%%%%%%%%%%%%  BEGIN DOC   %%%%%%%%%%%%%%%%%%%%%%%%%%%%%%%
\begin{document}

%%% TIKZ %%% 
\tikzstyle{every picture}+=[remember picture]
%\everymath{\displaystyle}

\tikzset{   
	every picture/.style={remember picture,baseline},
	every node/.style={anchor=base,align=center,outer sep=1.5pt},
	every path/.style={thick},
}
\newcommand\marktopleft[1]{%
	\tikz[overlay,remember picture] 
	\node (marker-#1-a) at (-.3em,.3em) {};%
}
\newcommand\markbottomright[2]{%
	\tikz[overlay,remember picture] 
	\node (marker-#1-b) at (0em,0em) {};%
}
\tikzstyle{every picture}+=[remember picture] 
\tikzstyle{mybox} =[draw=black, very thick, rectangle, inner sep=10pt, inner ysep=20pt]
\tikzstyle{fancytitle} =[draw=black,fill=red, text=white]


\addtocounter{framenumber}{-1}
\thispagestyle{empty}
\maketitle 
\newpage




%%%%%%%%%%%%%%%%%%%%%%%%%%  SLIDE   %%%%%%%%%%%%%%%%%%%%%%%%%%%%%%%%
\begin{frame}{Outline}
\thispagestyle{empty}
\addtocounter{framenumber}{-1}

Part 1: Theory
\begin{enumerate}
\item Demand for capital

\item User cost of capital

\item Adjustment costs

\item Tobin's Q
\end{enumerate}

\vspace{5mm}
Part 2: Empirical regularities 
\begin{enumerate}
\item Tobin's Q in the data
\item Investment sensitivity to cash flows
\item Zwick and Mahon (2017)
\end{enumerate}

\end{frame}


%%%%%%%%%%%%%%%%%%%%%%%%%%  SLIDE   %%%%%%%%%%%%%%%%%%%%%%%%%%%%%%%%
\begin{frame}{Overview}
\begin{witemize}
\item Investment is important for macroeconomics

\item Investment increases the stock of capital, therefore key determinant of growth

\item Investment is volatile, matters a lot for business cycle fluctuations

\end{witemize}
\end{frame}


%%%%%%%%%%%%%%%%%%%%%%%%%%  SLIDE   %%%%%%%%%%%%%%%%%%%%%%%%%%%%%%%%
\begin{frame}{1. Demand for capital}
\begin{witemize}
\item Consider a firm that operates a production function 
\begin{equation*}
	y_t = F(z_t, k_t, x_t)
\end{equation*}
where $z_t$ is exogenous productivity and $x_t$ other inputs

\item Suppose firm can rent capital \textit{frictionlessly} at rate $r_t^k$

\item Firm problem therefore given by
\begin{equation*}
	\max_k F(z_t, k_t, x_t^*) - r_t^k k_t
\end{equation*}
where $x_t^*$ denotes optimal choice of other inputs 

\item Optimal capital demand then determined by:
\begin{equation*}
	F_{k, t} \equiv \partial_k F(\cdot) = r_t^k
\end{equation*}

\end{witemize}
\end{frame}


%%%%%%%%%%%%%%%%%%%%%%%%%%  SLIDE   %%%%%%%%%%%%%%%%%%%%%%%%%%%%%%%%
\begin{frame}{2. User cost of capital}
\begin{witemize}
\item Capital usually not rented but owned by firms

\item What is the appropriate notion of ``rental rate''? $\implies$ user cost literature

\item Consider the deterministic firm (sequence) problem:
\begin{equation*}
	V(k_0) = \max_{ \{i_t\}_{t \geq 0} } \int_0^\infty e^{- \int_0^t r_s ds} \Big( f_t(k_t) - p_t i_t \Big) dt
\end{equation*}
where $f_t(k_t) = F(z_t, k_t, x_t^*)$, facing the capital accumulation technology 
\begin{equation*}
	\dot k_t = i_t - \delta k_t
\end{equation*}

\item Interpretation? (PE, small firm, take prices as given, SDF, ...)

\end{witemize}
\end{frame}


%%%%%%%%%%%%%%%%%%%%%%%%%%  SLIDE   %%%%%%%%%%%%%%%%%%%%%%%%%%%%%%%%
\begin{frame}{}
\begin{witemize}
\item How do we solve this problem? (i) optimal control theory (ii) dynamic programming 

\item Hamiltonian is: $\mathcal H_t (k_t, i_t, \lambda_t) = f_t(k_t) - p_t i_t + \lambda_t(i_t - \delta k_t)$ 

\item What are control and state? Cookbook solution:
	\begin{align*}
		0 &= \partial_i \mathcal H_t = - p_t + \lambda_t \\
		r_t \lambda_t - \dot \lambda_t &= \partial_k \mathcal H_t = f_{k, t} - \delta \lambda_t 
	\end{align*}

\item Practice: derive the transversality condition at home using calculus of variations

\item Combining and using the capital demand condition $r_t^k = f_{k, t}$:
\begin{equation*}
	\text{user cost } = r_t^k = \bigg( r_t + \delta - \frac{\dot p_t}{p_t} \bigg) p_t
\end{equation*}

\end{witemize}
\end{frame}


%%%%%%%%%%%%%%%%%%%%%%%%%%  SLIDE   %%%%%%%%%%%%%%%%%%%%%%%%%%%%%%%%
\begin{frame}{}
\begin{witemize}
\item Next: dynamic programming. Convince yourself you can derive HJB:
\begin{equation*}
	r_t V_t(k) = \partial_t V_t(k) + \max_i \bigg\{ f_t(k) - p_t i + (i - \delta k) \partial_k V_t(k) \bigg\}
\end{equation*}

\item Why is this HJB not stationary, i.e., why $\partial_t V_t(k)$?

\item FOC: $p_t = \partial_k V_t(k)$ $\implies$ price of investment = marginal value of capital

\end{witemize}
\end{frame}

%%%%%%%%%%%%%%%%%%%%%%%%%%  SLIDE   %%%%%%%%%%%%%%%%%%%%%%%%%%%%%%%%
\begin{frame}{}

\begin{equation*}
	r_t V_t(k) = \partial_t V_t(k) + \max_i \bigg\{ f_t(k) - p_t i + (i - \delta k) \partial_k V_t(k) \bigg\}
\end{equation*}

\begin{witemize}
\item Envelope condition: 
\begin{equation*}
	r_t \partial_k V_t(k) = \partial_{kt} V_t(k) + \partial_k f_t(k) - \delta \partial_k V_t(k) + (i(k) - \delta k) \partial_{kk} V_t(k)
\end{equation*}

\item Next, differentiate FOC $p_t = \partial_k V_t(k)$ with respect to time: 
\begin{equation*}
	\dot p_t = \frac{d}{dt} p_t = \frac{d}{dt} \partial_k V_t(k) = \partial_{kt} V_t(k) \frac{dt}{dt} + \partial_{kk} V_t(k) \frac{dk}{dt} 
\end{equation*}

\item Finally, put all together:
\begin{align*}
	& r_t \partial_k V_t(k) = \partial_k f_t(k) - \delta \partial_k V_t(k) + \dot p_t \\
	\implies & \text{user cost } = r_t^k = \partial_k f_t(k) = \bigg( r_t + \delta - \frac{\dot p_t}{p_t} \bigg) p_t
\end{align*}
\end{witemize}
\end{frame}


%%%%%%%%%%%%%%%%%%%%%%%%%%  SLIDE   %%%%%%%%%%%%%%%%%%%%%%%%%%%%%%%%
\begin{frame}{Interpreting user cost model}
\begin{witemize}
\item User cost of capital
\begin{itemize}
	\item increases with $r_t$
	\item increases with depreciation rate $\delta$
	\item decreases with capital gains rate $\dot p_t / p_t$
\end{itemize}

\item Hall and Jorgenson (1967): user cost model helpful to evaluate tax policies

\item But not very helpful for investment dynamics:
\begin{itemize}
	\item Model determines capital stock, change in $r_t^k$ requires `jumps'

	\item Decisions about capital stock become static, not forward-looking
\end{itemize}

\end{witemize}
\end{frame}


%%%%%%%%%%%%%%%%%%%%%%%%%%  SLIDE   %%%%%%%%%%%%%%%%%%%%%%%%%%%%%%%%
\begin{frame}{}
\begin{witemize}
\item High elasticity of capital demand to user cost of capital $\implies$ observed variation in interest rates would generate counter-factually large investment volatility

\item What might slow down the adjustment of the capital stock?

\item Internal adjustment costs:
\begin{itemize}
	\item Direct costs faced by firms 
	\item More costly construction and training of workers
	\item Disruption of current production
\end{itemize}

\item External adjustment costs:
\begin{itemize}
	\item Financing needs $\implies$ large upfront investment costs
	\item Capital goods distinct from consumption goods, distinct capital producing sector, firm may not be ``small''
\end{itemize}

\end{witemize}
\end{frame}


%%%%%%%%%%%%%%%%%%%%%%%%%%  SLIDE   %%%%%%%%%%%%%%%%%%%%%%%%%%%%%%%%
\begin{frame}{3. A model of firm investment with adjustment costs} 
\begin{witemize}
\item Let's start with simple quadratic (smooth + convex) adjustment cost:
\begin{equation*}
	C(i_t) , \quad \quad \text{ where } C(0) = 0, C'(0) = 0, C''(i_t) > 0
\end{equation*}


\item Firms problem now: 
\begin{equation*}
	V(k_0) = \max_{ \{i_t\}_{t \geq 0} } \int_0^\infty e^{- \int_0^t r_s ds} \bigg( f_t(k_t) - i_t - C(i_t) \bigg) dt
\end{equation*}
facing the capital accumulation technology 
\begin{equation*}
	\dot k_t = i_t - \delta k_t
\end{equation*}

\end{witemize}
\end{frame}


%%%%%%%%%%%%%%%%%%%%%%%%%%  SLIDE   %%%%%%%%%%%%%%%%%%%%%%%%%%%%%%%%
\begin{frame}{}
\begin{witemize}
\item HJB is given by (make sure this makes sense to you): 
\begin{equation*}
	r_t V_t(k) = \partial_t V_t(k) + \max_i \bigg\{ f_t(k) - i - C(i) + (i - \delta k) \partial_k V_t(k) \bigg\}
\end{equation*}

\item Implies FOC:
\begin{equation*}
	1 + C'(i_t(k)) = \partial_k V_t(k) 
	\quad \implies \quad
	i_t(k) = (C')^{-1} \Big( \partial_k V_t(k) - 1 \Big)
\end{equation*}

\item Envelope condition (dropping $t$ subscripts and abbreviating derivatives):
\begin{equation*}
	rV_k = V_{tk} + f_k + (i-\delta k) V_{kk} - \delta V_k
\end{equation*}


\end{witemize}
\end{frame}


%%%%%%%%%%%%%%%%%%%%%%%%%%  SLIDE   %%%%%%%%%%%%%%%%%%%%%%%%%%%%%%%%
\begin{frame}{}
\begin{witemize}
\item Rewriting:
\begin{equation*}
	(r - \delta) V_k = f_k + V_{tk} + (i-\delta k) V_{kk} 
\end{equation*}

\item Now again differentiate: $\dot V_k = V_{kt} + V_{kk} \dot k = V_{tk} + (i-\delta k) V_{kk}$, so
\begin{equation*}
	(r - \delta) V_k = f_k + \dot V_k 
\end{equation*}

\item We now introduce new notation:
\begin{equation*}
	\text{Tobin's (marginal) Q } = q = V_k
\end{equation*}

\item Firm's decision problem summarized by system of equations:
\begin{align*}
	(r - \delta) q &= f_k + \dot q \\
	\dot k &= i - \delta k \\
	C'(i) &= q - 1
\end{align*}

\end{witemize}
\end{frame}



%%%%%%%%%%%%%%%%%%%%%%%%%%  SLIDE   %%%%%%%%%%%%%%%%%%%%%%%%%%%%%%%%
\begin{frame}{}
\begin{witemize}

\item Denote the function:  
\begin{align*}
	i = (C')^{-1} ( q - 1 ) 
	\quad \implies \quad
	i = \Phi(q-1)
\end{align*}

\item Given initial capital $k_0$, sequences $\{q_t, k_t\}_{t \geq 0}$ solve the firm problem if they satisfy
\begin{align*}
	\dot q_t &= (r_t - \delta) q_t - f_{k, t} \\
	\dot k_t &= \Phi(q_t-1) - \delta k_t
\end{align*}

\item System of 2 ODEs. $q$ equation is \textit{forward-looking} while $k$ equation is \textit{backward-looking}. Why?

\item We can solve $q$ equation forward to obtain:
\begin{equation*}
	q_t = \int_t^\infty e^{- \int_t^s (r_\ell - \delta) d\ell} f_{k, s} ds 
\end{equation*}

\end{witemize}
\end{frame}


%%%%%%%%%%%%%%%%%%%%%%%%%%  SLIDE   %%%%%%%%%%%%%%%%%%%%%%%%%%%%%%%%
\begin{frame}{}
\begin{witemize}

\item Analyze via phase diagram, suppose $r_t = r$ and $\delta = 0$ 

\item Steady state: $q = 1$ and $r = f_k$

\item $q_t$ is a jump variable (like consumption)

\item Capital $k_t$ is a predetermined state variable, must always follow continuous path (infinite investment would be too costly)

\end{witemize}
\end{frame}


%%%%%%%%%%%%%%%%%%%%%%%%%%  SLIDE   %%%%%%%%%%%%%%%%%%%%%%%%%%%%%%%%
\begin{frame}{4. Tobin's Q}
\begin{witemize}
\item What is the significance of $q_t = \partial_k V_t(k)$? 

\item $q_t$ is the (shadow) value of one additional unit of installed capital

\item Connection between sequence problem and dynamic programming: $q_t$ turns out to be the Lagrange multiplier on relaxing capital / investment constraint

\item In this model, $q_t$ is a sufficient statistic for firm investment 

\item Firms invest whenever shadow value of installed capital is larger than value of consumption good: $q_t > 1$

\item Without adjustment costs, firms invest until $q_t = 1$

\item With, firms invest until excess value equal to adjustment cost on margin
\end{witemize}
\end{frame}


%%%%%%%%%%%%%%%%%%%%%%%%%%  SLIDE   %%%%%%%%%%%%%%%%%%%%%%%%%%%%%%%%
\begin{frame}{}
\begin{witemize}
\item In the data, \textbf{average Q} easier to measure than \textbf{marginal Q} 

\item Our theory: marginal Q is sufficient statistic for investment

\item Tobin (1969) argued that firms should invest if 
\begin{equation*}
	Q = \frac{ \text{Market value of firm capital}}{\text{Book value of capital}} > 1
\end{equation*}

\item Hayashi (1982): Average Q = marginal Q when markets competitive and production function + adjustment cost homogeneous of degree 1

\end{witemize}
\end{frame}


%%%%%%%%%%%%%%%%%%%%%%%%%%  SLIDE   %%%%%%%%%%%%%%%%%%%%%%%%%%%%%%%%
\begin{frame}{5. Tobin's Q and investment in the data}
\begin{witemize}
\item Q-theory suggests Q (NPV of marginal projects avaialble to the firm) is sufficient statistic

\item Under additional conditions, average Q (market relative to book value) encodes the NPV of these marginal projects

\item Other variables such as contemporaneous cash flows should not matter. We can test this 

\item Suppose we estimate regression:
\begin{equation*}
	\frac{I_{it}}{K_{it}} = \alpha + \beta Q_{it} + \epsilon_{it}	
\end{equation*}

\item Summers (1981, Brookings) estimates this by OLS and finds $\beta = 0.031 (0.005)$. Very low, implies high adjustment cost!

\end{witemize}
\end{frame}


%%%%%%%%%%%%%%%%%%%%%%%%%%  SLIDE   %%%%%%%%%%%%%%%%%%%%%%%%%%%%%%%%
\begin{frame}{}
\begin{witemize}
\item Suppose we estimate regression:
\begin{equation*}
	\frac{I_{it}}{K_{it}} = \alpha + \beta Q_{it} + \epsilon_{it}	
\end{equation*}

\item Problem:
\begin{itemize}
	\item Where did the $\epsilon_{it}$ come from?
	\item Q-theory says that there should be no $\epsilon_{it}$
	\item To estimate this equation, we need to know about $\epsilon_{it}$. Is it orthogonal to $Q_{it}$?!
\end{itemize}

\item View 1: measurement error in $Q_{it}$:
\begin{itemize}
	\item Stocks are volatile and may not reflect fundamental value
	\item Assumptions under which marginal = average Q may not hold
	\item Measurement error would result in attenuation bias
\end{itemize}

\item View 2: model is wrong and other factors affect investment
\begin{itemize}
	\item If other factors that raise desired investment also raise interest rates, they may lower $Q_{it}$ and downward bias $\beta$
	\item Bias could go either way 
\end{itemize}
\end{witemize}
\end{frame}


%%%%%%%%%%%%%%%%%%%%%%%%%%  SLIDE   %%%%%%%%%%%%%%%%%%%%%%%%%%%%%%%%
\begin{frame}{Beyond Q}
\begin{witemize}
\item Suppose we are interested in whether internal funds (cash flows) affect investment. Why? 

\item Simple approach:
\begin{equation*}
	\frac{I_{it}}{K_{it}} = \alpha + \beta_1 Q_{it} + \beta_2 \frac{CF_{it}}{K_{it}} + \epsilon_{it}
\end{equation*}

\item Problem:
\begin{itemize}
	\item Cash flow likely correlated with future profitability
	\item If $Q_{it}$ is mismeasured, cash flow would proxy for Q even if financial markets are perfect
\end{itemize}

\end{witemize}
\end{frame}


%%%%%%%%%%%%%%%%%%%%%%%%%%  SLIDE   %%%%%%%%%%%%%%%%%%%%%%%%%%%%%%%%
\begin{frame}{Fazzari-Hubbard-Petersen (1988)}
\begin{witemize}
\item Use diff-in-diff strategy to circumvent cash flow - profitability correlation problem

\item Different groups of firms: e.g. high- vs. low-dividend-firms

\item Low dividends proxy for greater financial constraints

\item Is investment more sensitive to cash flows for low-dividend?

\item Identifying assumption: cash flow - profitability correlation the same on average for two groups

\item \textbf{Result:} Cash flow sensitivity much higher for low dividend firms. Marginal propensity to invest out of cash flows very high. 

\end{witemize}
\end{frame}


%%%%%%%%%%%%%%%%%%%%%%%%%%  SLIDE   %%%%%%%%%%%%%%%%%%%%%%%%%%%%%%%%
\begin{frame}{Do cash flows matter for investment?}
\begin{witemize}
\item Ideal: find a shock to cash flows that is orthogonal to investment opportunities (Q)

\item Well-known examples:
\begin{itemize}
	\item Lamont (1997): investment of non-oil subsidiaries of oil companies falls when oil prices fall

	\item Rouh (2006): investment of firms with underfunded pension plans due to drops in asset prices. Compare firms with and without mandatory contributions. Those with mandatory contributions see investment fall more.
\end{itemize}


\end{witemize}
\end{frame}


%%%%%%%%%%%%%%%%%%%%%%%%%%  SLIDE   %%%%%%%%%%%%%%%%%%%%%%%%%%%%%%%%
\begin{frame}{Temporary investment tax incentives: bonus depreciation}
\begin{witemize}
\item Suppose the government introduces a policy that changes the marginal benefit or cost of investment. This is arguably exogenous to the firm.

\item Firms pay taxes on income net of business expenses

\item Can fully expense wages, advertising, etc. immediately 

\item But investment gets expensed over time according to tax depreciation schedules

\item Bonus depreciation accelerates this depreciation schedule
\end{witemize}
\end{frame}


%%%%%%%%%%%%%%%%%%%%%%%%%%  SLIDE   %%%%%%%%%%%%%%%%%%%%%%%%%%%%%%%%
\begin{frame}{Zwick-Mahon (2017)}
\begin{witemize}
\item Bonus depreciation changes occur in recessions: may be correlated with other determinants of investment

\item Zwick-Mahon use diff-in-diff strategy:
\begin{itemize}
	\item Bonus more valuable in industries with longer lived investments
	\item Compare effects of bonus across industries with differing investment duration
\end{itemize}

\item Find large effects. More liquidity constrained firms have larger effects

\item Effect only exists for firms with immediate tax benefit

\end{witemize}
\end{frame}


\end{document}
